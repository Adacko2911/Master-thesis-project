In this thesis, we introduced the Multilevel Explicit Duration Hidden Markov model, a novel method to model hidden behavioural patterns with Intensive longitudinal data. We performed the Monte Carlo simulation study with varying levels of a number of states, state duration and observed sequence length in order to assess their effects on MEDHMM performance and compare it with the MHMM results under the same simulation scenarios. We explicitly aimed to answer the question if the more complex and computationally intensive MEDHMM can capture the true hidden state durations with greater accuracy than the MHMM and estimate the other model-specific parameters (i.e. group-level emission and gamma distribution parameters) with the same precision as the MHMM does. Bias, precision, and coverage were used to identify the performance of each model. Finally, to demonstrate the applicability of MEDHMM we applied the method to empirical Bipolar disorder data and show how it can be modelled to recover the bipolar disorder mood state dynamics for the intensive longitudinal study. 

\subsection{Main Findings}
The Monte Carlo study revealed that the MEDHMM can outperform the MHMM in dwell time estimation, as well as, in the estimation of other state-specific group-level parameters when state durations are greater or equal to $19.5$. On the other hand, the MEDHMM failed to recover the true emission distribution means for a shorter dwell time of $1.4$ and for some scenarios with state durations equal to $3.5$. Further, results showed the MEDHMM performed slightly better than the MHMM in state decoding accuracy for the longer state durations. The results from the additional scenario, with varying across states dwell times, showed that the MEDHMM performed well in the estimation of group-level parameters. This suggested that the MEDHMM is able to distinguish between states more effectively when state durations are not equal, which is in line with \cite{Ruiz_Suarez_2022}. We demonstrated that the MHMM could perform well in modelling the state-specific emission distributions overall. In addition, the MHMM was able to capture the shorter durations of states better. We also saw that the MHMM overestimated the self-transition probabilities and underestimated the switching between two different states probabilities. This result was in line with the existing research on Multilevel Hidden Markov models, as \cite{McClintock_2021} and \cite{jonsen2016joint} point out that estimating transition parameters which are closer to the parameters' space boundaries (i.e., 0 and 1) are more challenging to establish for MHMM. Nevertheless, the results from the adjusted transition matrix without diagonal entries confirmed that the MHMM can recover the true transition probabilities assumed by the simulation study well, however still not as good as the MEDHMM when longer dwell times are assumed. 

Having the main findings listed, we would like to further describe how each simulation factor influenced the estimates of the current study. 

We showed that with increasing dwell time, the performance of MEDHMM of estimation of all parameters improved significantly and results were more accurate than for MHMM. Our findings were in line with \cite{Ruiz_Suarez_2022} who investigated the effect of varying dwell time for single-level EDHMM and HMM. Similar to them, we found that the model decoding accuracy of MEDHMM exceeds the accuracy of the MHMM for longer states. Also in line with the results in \cite{Ruiz_Suarez_2022}, we saw that the dwell time estimation is more accurate when the assumed dwell time is long or differs within states. However, our conclusion is limited to one additional scenario with different state duration we investigated and further durations that are needed to establish it. For the dwell time distributions, we saw that for scenarios with assumed $1.4$ state duration, the MEDHMM was not able to estimate the dwell time means parameters accurately. The reason for that is the low density around $0$ values for the log-normal distribution, which make the distributions not suitable for modelling short time durations and cannot account for the inflated amount of state durations of $1$ (\citealp{hadjamar2022bayesian}). 

With the increase in the number of observations, parameter estimates from the MHMM exhibit marginally decreasing bias and increased precision and coverage (e.g., \citealp{McClintock_2021}). In terms of the MEDHMM parameter estimation, we observed the same tendencies. The results were expected as with the increasing amount of information the models had more power to estimate parameters of posterior distribution more accurately.  

The effect of increasing number of states was not as prominent as the effects for varying state durations and observation sequence lengths. On average, the coverage of group-level parameters was larger and the bias was smaller for the $3$ state model in comparison to the $4$ state scenarios for both methods. According to \cite{LUATI2021107183} for a single-level EDHMM the accuracy of the decoding should decrease with the number of states. Our results did not agree as for the MEDHMM we recorded improvements in decoding accuracy when comparing $3$ and $4$ state scenarios. 

\subsection{Limitations}
Nonetheless, the founding of our research must be interpreted with caution and some limitations should be borne in mind.
Firstly, the simulation was designed in such a way that the dwell times were equal over states. That way we were able to capture the effect of increasing dwell time, however, the effect of varying dwell times that may often accrue in real data was not assessed. The work of \cite{Ruiz_Suarez_2022} showed that the single-level EDHMM is able to capture scenarios with varying dwell times well, we could confirm the finding only through the extent of our additional simulation scenario where we assumed  different dwell times across 3 proposed states and 500 observations per subject. Nevertheless, more extensive research would have to be implemented in order to make the result valid for a wider range of possible scenarios. 

Secondly, we used the log-normal distribution to model the state durations in MEDHMM which appeared not an appropriate choice for modelling durations that were short and close to $1$. Multiple studies of single-level EDHMMs proposed different dwell time distributions. For example, \cite{hadjamar2022bayesian,dewar_inference_2012,LUATI2021107183} proposed Poisson and shifted Poisson distribution, \cite{hadjamar2022bayesian,POHLE2022107479} proposed Negative binomial distribution and \cite{Nagaraja_1996} used inverse gamma distribution. We would expect some of them to be a better fit for shorter durations or generalize well for a wider spectrum of state durations. That said, further research is required in order to implement and evaluated different distribution types for modelling intensive longitudinal data with the MEDHMM. 

Thirdly, our research showed that the simulated data were easy to solve for the MHMM as the simulation considered the state-dependent emission distributions to exhibit a low overlap, which was in line with \cite{McClintock_2021}. As a result, we could not access the effect of noisy or more imperfect states definitions. Generally, Hidden Markov models are known to perform poorly when the state-dependent distributions highly overlap (e.g. \cite{jonsen2016joint,Beyer_Morales_Murray_Fortin_2013}). The study of \cite{Ruiz_Suarez_2022} showed that the EDHMM can outperform the HMM when the overlap between state emission distributions is high, as the information about state durations allows the EDHMM to differentiate between states.  Because the HMM (and MHMM) rely solely on emission distributions to differentiate between states, it is expected that this model will fail when the emission distributions overlap. Those results, if applicable for MEDHMM, could introduce promising added value of recovering the emission distributions for noisy or highly overlapping data, as it is expected to experience not-so-well-defined states in empirical applications.

Lastly, we would like to point out that the execution models took considerably more time for the MEDHMM (23 hours on average) than the MEDHMM (2 hours on average). The significant difference between computational times was expected, as the structure of the forward-backwards algorithm for MEDHMM requires an additional step of calculating the most probable state duration for each value in range $1:d_{max}$. In MHMM, this calculation is not present, making the model more time efficient. Some studies of single-level EDHMM proposed different techniques to optimize the process, however, more investigation is required to improve the time of estimation. 

 

\subsection{Recommendations}
In this part, we summarized some recommendations for researchers planning to analyse ILD using MEDHMM or MHMM. \\\\
\emph{When focusing on emissions?}\\
If the researcher's focus is on obtaining the distributions of the observable data within established hidden states (emission distribution), the MHMM serves well independently of the duration of each state, the total number of states and the observation length. For longer state durations (i.e. d$\in{19.5, 99.5}$) the emission distribution parameters are closer to the truth, and we can expect that even for the observation length of 200 the variances perform better in terms of the bias than the MHMM.  \\ \\
\emph{When focusing on dwell times?}\\
If one wishes to recover the expected dwell times, there are a few options. When certain about the expected state persistence being longer than 20-time occasions, one should consider the use of the MEDHMM over the MHMM. In addition, we would always advise incorporating prior information about the dwell time distribution, which as shown in \cite{hadjamar2022bayesian} can lead to greater accuracy in dwell time parameters' estimation. Priors for expected state durations are not present in HMM. If the researcher is not sure about the expected dwell times, the less computationally intensive MHMM should be implemented as it gives the reader some guidelines about the expected state duration or at least the values can be used to establish more and less frequent states. \\ \\
\emph{When focusing on transition probabilities?}\\
We showed that MEDHMM recognizes the transition probabilities well, except in the scenario with the 1.4 dwell time. However, if one wishes to focus on the transitions between latent states, the adjusted transition probability matrix (excluding self-transitions) derived from MHMM results serves well in terms of recognizing the correct probabilities when the probabilities of the switch are equal. The computational time, especially for the longer observation sequences, might escalate quickly for the MEDHMM. 
We want to mention that our study did not test scenarios with unequal transition probabilities, hence the advice should be taken in light of the limitation. \\\\
\emph{When focusing on decoding?}\\ 
If one expects the dwell time to be longer than $3.5$ in light of our study, the decoding will be slightly more accurate for the MEDHMM. That said, if we would like to only focus on the decoding performance, the MHMM should be used as it can recognize both the state decoding with an overall accuracy of $\kappa>0.97$. The cautious however needs to be taken as in the current study we only inspected well-separated emission distributions and that's why the MHMM algorithm could use the estimates to correctly assess states. When the emission distributions would overlap a lot, the accuracy of the MEDHMM is expected to increase, as shown in \cite{Ruiz_Suarez_2022}. 


