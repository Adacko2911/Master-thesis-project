Human behaviours and disorders are characterised by evolving patterns of events that affect our physical and mental well-being. Therefore, it is essential for researchers to identify and classify relevant behaviours and model their dynamic processes to fully understand the origins of human conditions, discover new treatments, and develop preventative strategies. These methods require collecting multiple time points per individual at a high frequency over time, which has only recently become possible for scientists thanks to new technological advances \citep{shiffman_ecological_2008,ARIENS2020110191}. The high-frequency data, also known as the \ac{ild}, can be collected with methods such as ecological momentary assessment, daily diaries, or ambulatory assessment. It contains detailed information about individuals and has prompted the development of models that allow for the investigation of human-based processes \citep{bolger2013intensive}.

Multilevel approaches, in particular, are effective at capturing the complexities of human actions for a group of people \citep[][]{Walls_Schafer_2006}. These methods account for both homogeneity of observations within, and heterogeneity between individuals, hence researchers often use the models to investigate individual changes in behavioural patterns and reflect them across the entire research group \citep{hamaker_no_2017}. Some humane-related conditions can be summarised with a certain number of discrete states that alternate over time, for example, burn-out and enthusiasm in burn-out condition \citep{Maslach_Leiter_2016}, manic and depressive states in bipolar disorder \citep{Kukopulos_Reginaldi_Laddomada_Floris_Serra_Tondo_2008} or binge-eating and normal eating in an eating disorder \citep{Bulik_Butner_Tregarthen_Thornton_Flatt_Smith_Carroll_Baucom_Deboeck_2020}. Such behavioural conditions are complex concepts that are frequently impossible to measure due to a lack of direct quantifiers. Thus, in order to investigate the development of such behaviours, scientists seek models that not only accommodate multilevel structures to account for individuality factors but also allow for the modelling of discrete and latent states that change over time and can be quantified using measurable indicators.

A probabilistic method that meets the above requirements is the \ac{mhmm} introduced by \cite{altman_mixed_2007}. The \ac{mhmm} captures the relationships between the latent discrete states and sequential observations by fitting the \ac{ild} to a multilevel framework to obtain individual trajectories of behavioural patterns \citep[e.g.,][]{Baum_Eagon_1967, Rabiner_1989, Zucchini_MacDonald_2009}. The \ac{hmm} and its multilevel extensions have been widely applied in statistics and other research areas, such as speech recognition \citep{Markov_1913,Juang_Rabiner_1991}, bioinformatics \citep{Durbin_1998,Eddy_1998}, ecology \citep{Joo_Bertrand_Tam_Fablet_2013,McClintock_2021}, humane activity recognition \citep{Czúni_2010,Ronao_Cho_2017,martindale_smart_2017,cohen_emo}, animal activity \citep{Bode_Seitz_2018,McKellar_Langrock_Walters_Kesler_2015}, brain activity \citep{kirchherr_bayesian_2022,El_Moursy_ElAzhary_Younis_2014} or labelling DNA sequences \citep{Karplus_Barrett_Hughey_1998} and many more. While well-adapted for inferring latent processes from visible behavioural events \citep{Boussemart}, the \ac{hmm} implicitly assumes geometrically distributed time spent in latent behavioural states (dwell time). Consequently, shorter dwell times always appear more likely than longer ones, which sometimes might not be a good representation for behavioural data \citep{Tokdar_Xi_Kelly_Kass_2010,Emm_Th}. For example, in bipolar disorder modelled with \ac{hmm}, remaining in a depressive state for one hour would be always more probable than remaining in the state for 12 or 24 hours. Since the geometric distribution might not be a good representation of real dwell times, a less restricted version of the \ac{hmm} is needed. 

One approach to overcome the limitation of the dwell time modelled with geometric distribution is the \acl{edhmm} \cite[EDHMM; ][]{freguson1980variable} which is a specific type of the \acl{hsmm}\footnote{An extensive review of Semi-Hidden Markov models, including the explicit-duration method, can be found in \cite{murphy_hidden_nodate} and \cite{Yu_2010}.} (HSMM). This method is a generalization of the \ac{hmm} that can explicitly capture relationships among the latent states, state durations and sequential observations. Moreover, the \ac{edhmm} was shown superior in dwell times estimation in studies of \cite{Shappell_Caffo_Pekar_Lindquist_2019,Ruiz_Suarez_2022}. Since its introduction, \ac{hsmm}s have been applied in numerous human-related research areas, such as speech recognition/synthesis \citep{russell1985explicit,Juang_Rabiner_1991}, human activity recognition/prediction \citep{4118806,duong2005activity,van2010activity}, functional MRI brain mapping \citep{faisan2002hidden,1388575}, and network anomaly detection \citep{duong2005activity,Tan_Xi_2008,Bang_Cho_Kang_2017}. The multilevel extensions of the \ac{medhmm}, on the other hand, are a novel concept and to our knowledge, no articles on this topic have been published till now. Knowing that \ac{edhmm} and its existing extensions are computationally demanding \citep[][]{mitchell_complexity_1995,johnson_bayesian_nodate} and no easy-to-use software exists to apply them, it is still unclear whether the model would be a good fit for the behavioural data and whether it would actually outperform the MHMM in terms of state duration modelling.

However, given its potential to capture the reflection of behavioural state durations more effectively, we hypothesise that the \ac{medhmm} would be preferable to \ac{mhmm} in the context of behavioural process modelling. In order to evaluate the hypothesis, we present the first, simulation study comparing the \ac{medhmm} with the \ac{mhmm} by simulating data from the \ac{medhmm} and fitting it to both methods using Bayesian estimation. We wish to inspect if more complex and computationally intensive \ac{medhmm} can capture the true hidden state durations with greater accuracy than the \ac{mhmm}. Also, we want to evaluate if \ac{medhmm} is capable of estimating other model-specific parameters with the same precision as the \ac{mhmm} does. We assess the models' performance in various conditions mimicking the behavioural data by varying three design levels: 1) mean dwell time, since we expect the \ac{medhmm} to be able to capture longer duration more accurately and short with a similar precision compared to the \ac{mhmm}; 2) the number of hidden states as with more states we introduce more complexity to the model; \linebreak 3) sample size, since in general, the model inference depends on the length of the subject-specific sequence of observations.

The rest of the thesis is organized as follows. We give an overview of \ac{hmm} and \ac{edhmm}; we introduce the structure of the models and describe how they can be extended to the multilevel framework (MHMM and MEDHMM) and applied in the context of continuous \ac{ild}. We clarify how assumptions of the \ac{mhmm} can be relaxed in order to construct the \ac{medhmm}. Next, the Monte Carlo simulation study is conducted to compare the performance of both models under various sampling conditions in order to characterize when and if applying the \ac{medhmm} would be beneficial in the context of behavioural \ac{ild} for continuous outcomes. Finally, we analyse an empirical dataset from a study of bipolar disorder with \ac{medhmm} to showcase how the method can be applied to behavioural research dilemmas. We conclude with a discussion and possible future directions. 

